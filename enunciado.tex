\documentclass[letter, 11pt]{article}
%% ================================
%% Packages =======================
\usepackage[utf8]{inputenc}      %%
\usepackage[T1]{fontenc}         %%
\usepackage{lmodern}             %%
\usepackage[spanish]{babel}      %%
\decimalpoint                    %%
\usepackage{fullpage}            %%
\usepackage{fancyhdr}            %%
\usepackage{graphicx}            %%
\usepackage{amsmath}             %%
\usepackage{color}               %%
\usepackage{mdframed}            %%
\usepackage[colorlinks]{hyperref}%%
%% ================================
%% ================================

%% ================================
%% Page size/borders config =======
\setlength{\oddsidemargin}{0in}  %%
\setlength{\evensidemargin}{0in} %%
\setlength{\marginparwidth}{0in} %%
\setlength{\marginparsep}{0in}   %%
\setlength{\voffset}{-0.5in}     %%
\setlength{\hoffset}{0in}        %%
\setlength{\topmargin}{0in}      %%
\setlength{\headheight}{54pt}    %%
\setlength{\headsep}{1em}        %%
\setlength{\textheight}{8.5in}   %%
\setlength{\footskip}{0.5in}     %%
%% ================================
%% ================================

%% =============================================================
%% Headers setup, environments, colors, etc.
%%
%% Header ------------------------------------------------------
\fancypagestyle{firstpage}
{
  \fancyhf{}
  \lhead{\includegraphics[height=4.5em]{LogoDFI.jpg}}
  \rhead{FI3104-1 \semestre\\
         Métodos Numéricos para la Ciencia e Ingeniería\\
         Prof.: \profesor}
  \fancyfoot[C]{\thepage}
}

\pagestyle{plain}
\fancyhf{}
\fancyfoot[C]{\thepage}
%% -------------------------------------------------------------
%% Environments -------------------------------------------------
\newmdenv[
  linecolor=gray,
  fontcolor=gray,
  linewidth=0.2em,
  topline=false,
  bottomline=false,
  rightline=false,
  skipabove=\topsep
  skipbelow=\topsep,
]{ayuda}
%% -------------------------------------------------------------
%% Colors ------------------------------------------------------
\definecolor{gray}{rgb}{0.5, 0.5, 0.5}
%% -------------------------------------------------------------
%% Aliases ------------------------------------------------------
\newcommand{\scipy}{\texttt{scipy}}
%% -------------------------------------------------------------
%% =============================================================
%% =============================================================================
%% CONFIGURACION DEL DOCUMENTO =================================================
%% Llenar con la información pertinente al curso y la tarea
%%
\newcommand{\tareanro}{3}
\newcommand{\fechaentrega}{11/10/2018 23:59 hrs}
\newcommand{\semestre}{2018B}
\newcommand{\profesor}{Valentino González}
%% =============================================================================
%% =============================================================================


\begin{document}
\thispagestyle{firstpage}

\begin{center}
  {\uppercase{\LARGE \bf Tarea \tareanro}}\\
  Fecha de entrega: \fechaentrega
\end{center}


%% =============================================================================
%% ENUNCIADO ===================================================================
\noindent{\large \bf Problema 1}

El objetivo de este problema es investigar cómo se comporta la interpolación
con polinomios versus la interpolación spline en algunos casos interesantes.

Considere la función de Runge

$$ f(x) = \frac{1}{1 + 25 x^2} $$

\noindent en el intervalo $[-1, 1]$. Divida el intervalo en 4 tramos
equiespaciados (es decir, samplee 5 puntos en el intervalo $[-1, 1]$). Ahora
interpole un polinomio (usando, por ejemplo, el método de Lagrange) que pase
por esos 5 puntos. Haga lo mismo usando una interpolación spline.

Ahora aumente secuencialmente el número de puntos y compruebe cómo se comportan
los dos métodos (mejoran?, empeoran?, es lo que esperaba?).

\begin{ayuda}
  \noindent {\bf Nota.}

  Puede programar su propio método de Lagrange y/o spline, o puede utilizar
  alguna librería que le parezca adecuada. Investigue, por ejemplo, el módulo
  de interpolación de \texttt{scipy}. Si decide usar una librería, asegúrese de
  entender los detalles de las implementaciones (¿qué pasa en los extremos de
  la interpolación \texttt{spline}, por ejemplo?). Incluya esta información en
  el informe.
\end{ayuda}
%% FIN ENUNCIADO ===============================================================
%% =============================================================================



\end{document}
